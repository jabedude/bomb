\documentclass{article}
\usepackage[margin=1in]{geometry}
\usepackage{listings}
\usepackage{color}

\lstdefinelanguage
   [x64]{Assembler}     % add a "x64" dialect of Assembler
   [x86masm]{Assembler} % based on the "x86masm" dialect
   % with these extra keywords:
   {morekeywords={CDQE,CQO,CMPSQ,CMPXCHG16B,JRCXZ,LODSQ,MOVSXD, %
                  POPFQ,PUSHFQ,SCASQ,STOSQ,IRETQ,RDTSCP,SWAPGS, %
                  rax,rdx,rcx,rbx,rsi,rdi,rsp,rbp, %
                  r8,r8d,r8w,r8b,r9,r9d,r9w,r9b, %
                  r10,r10d,r10w,r10b,r11,r11d,r11w,r11b, %
                  r12,r12d,r12w,r12b,r13,r13d,r13w,r13b, %
                  r14,r14d,r14w,r14b,r15,r15d,r15w,r15b}} % etc.

\lstset{language=[x64]Assembler}
\definecolor{codegreen}{rgb}{0,0.6,0}
\definecolor{codegray}{rgb}{0.5,0.5,0.5}
\definecolor{codepurple}{rgb}{0.58,0,0.82}
\definecolor{backcolour}{rgb}{0.95,0.95,0.92}
 
\lstdefinestyle{mystyle}{
    backgroundcolor=\color{backcolour},   
    commentstyle=\color{codegreen},
    keywordstyle=\color{magenta},
    numberstyle=\tiny\color{codegray},
    stringstyle=\color{codepurple},
    basicstyle=\footnotesize,
    breakatwhitespace=false,         
    breaklines=true,                 
    captionpos=b,                    
    keepspaces=true,                 
    numbers=left,                    
    numbersep=5pt,                  
    showspaces=false,                
    showstringspaces=false,
    showtabs=false,                  
    tabsize=2
}
 
\lstset{style=mystyle}


\begin{document}

\begin{titlepage}
	\centering
	{\scshape\LARGE Tool Developer Qualification Course\par}
	\vspace{1cm}
	{\scshape\Large Reverse Engineering project\par}
	\vspace{1.5cm}
	{\huge\bfseries Bomb\par}
	\vspace{2cm}
	{\Large\itshape Joshua Abraham\par}
	\vfill
	Taught by by\par
	Dr. Joseph \textsc{Santmeyer}
	\vfill
	{\large \today\par}
\end{titlepage}


\newpage
\tableofcontents
\newpage


\section{Summary}
\begin{flushleft}
\vspace{.5pc}
\end{flushleft}
\par
The Fibonacci project has been a challenging experience.  Lack of high-level
loops, conditionals, etc. made implementing a program that calculates the 
Fibonacci sequence hard. Having to calculate up to the 500\textsuperscript{th}
number (an absurdly large value) made it even harder.  However, this project 
was a resounding success.  All requirements were accomplished, with minimal
C function calls.  The program is fast, easy to read, and well documented.  I
learned a great deal during development; one of the most practical lessons was
being able to translate high-level C into assembly almost line by line.  If I 
had more time, I would spend it on removing my call to \textit{printf} 
entirely and use 100\% assembly.

\begin{enumerate}
  \item Stage 1: swordfish
  \item Stage 2: jabraham
  \item Stage 3: 1872
  \item Stage 4: 107 214 428 856 1713
  \item Stage 5: ~ 1172 ~
  \item Stage 6: 48 a 48
  \item Stage 7: (press enter)
  \item Stage 8: (run ./stag8.sh in the same directory as the binary, then 
  press enter).
\end{enumerate}

\newpage
\section{Stage 1 Analysis}
\begin{flushleft}
\vspace{.5pc}
\end{flushleft}
\par
Stage 1 is straightforward.  Below is an abbreviated listing of the relevant
disassembled code:
\begin{lstlisting}
mov     esi, 0x4018f8   ; "swordfish"
mov     rdi, rax        ; user input
call    strcmp
test    eax, eax
setz    al
leave
\end{lstlisting}
\par
This stage calls strcmp with user input and the string "swordfish".  If the 
return value is zero (meaning the strings are identical), this stage returns 
non-zero and the program proceeds.

\newpage
\section{Stage 2 Analysis}
\begin{flushleft}
\vspace{.5pc}
\end{flushleft}
\par
Stage 2 starts with a call to getenv.  This function takes one parameter and 
returns the value of a specified environment variable.
\begin{lstlisting}
mov     edi, 0x401902    ; "USER"
call    getenv
mov     QWORD PTR [rip+0x201264], rax
\end{lstlisting}
\par
The user name is stored in a global variable that is used throughout the 
binary at rip+0x201264.  In my case the user name is "jabraham".  This and the 
saved user input are passed to strcmp, similarly to Stage 1.
\begin{lstlisting}
mov     rax, [rbp-0x8]
mov     rsi, rdx        ; "jabraham"
mov     rdi, rax        ; "jabraham"
call    strcmp
test    eax, eax
setz    al
leave
\end{lstlisting}
\par
After the call to strcmp, the program checks the return value in eax and sets 
al if the zero flag is set.
\end{document}}