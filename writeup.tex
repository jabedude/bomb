\documentclass{article}
\usepackage[margin=1in]{geometry}
\usepackage{listings}
\usepackage{color}

\lstdefinelanguage
   [x64]{Assembler}
   [x86masm]{Assembler}
   {morekeywords={CDQE,CQO,CMPSQ,CMPXCHG16B,JRCXZ,LODSQ,MOVSXD, %
                  POPFQ,PUSHFQ,SCASQ,STOSQ,IRETQ,RDTSCP,SWAPGS, %
                  rax,rdx,rcx,rbx,rsi,rdi,rsp,rbp, %
                  r8,r8d,r8w,r8b,r9,r9d,r9w,r9b, %
                  r10,r10d,r10w,r10b,r11,r11d,r11w,r11b, %
                  r12,r12d,r12w,r12b,r13,r13d,r13w,r13b, %
                  r14,r14d,r14w,r14b,r15,r15d,r15w,r15b}} % etc.

\lstset{language=[x64]Assembler}
\definecolor{codegreen}{rgb}{0,0.6,0}
\definecolor{codegray}{rgb}{0.5,0.5,0.5}
\definecolor{codepurple}{rgb}{0.58,0,0.82}
\definecolor{backcolour}{rgb}{0.95,0.95,0.92}
 
\lstdefinestyle{mystyle}{
    backgroundcolor=\color{backcolour},   
    commentstyle=\color{codegreen},
    keywordstyle=\color{magenta},
    numberstyle=\tiny\color{codegray},
    stringstyle=\color{codepurple},
    basicstyle=\footnotesize,
    breakatwhitespace=false,         
    breaklines=true,                 
    captionpos=b,                    
    keepspaces=true,                 
    numbers=left,                    
    numbersep=5pt,                  
    showspaces=false,                
    showstringspaces=false,
    showtabs=false,                  
    tabsize=2
}
 
\lstset{style=mystyle}


\begin{document}

\begin{titlepage}
	\centering
	{\scshape\LARGE Tool Developer Qualification Course\par}
	\vspace{1cm}
	{\scshape\Large Reverse Engineering project\par}
	\vspace{1.5cm}
	{\huge\bfseries Bomb\par}
	\vspace{2cm}
	{\Large\itshape Joshua Abraham\par}
	\vfill
	Taught by by\par
	Dr. Joseph \textsc{Santmeyer}
	\vfill
	{\large \today\par}
\end{titlepage}


\newpage
\tableofcontents
\newpage


\section{Summary}
\begin{flushleft}
\vspace{.5pc}
\end{flushleft}
\par
I began my analysis by reading the assembly of each stage to get a general
idea of what they were doing.  This static analysis is my preferred way to
solve challenges like these.  Some of the assembly code snippets in this
document has been edited to be more readable (i.e. identifying local 
variables with a name like "input" instead of an rbp offset).
\begin{lstlisting}[language=bash]
gdb -batch -ex 'file ./bomb.patched' -ex 'disassemble /r stage_1' > objectdumps/stage_1.out
....
gdb -batch -ex 'file ./bomb.patched' -ex 'disassemble /r stage_8' > objectdumps/stage_8.out
\end{lstlisting}

\par
While performing dynamic analysis with the GNU Debugger (gdb), the program 
pretends to segfault by detecting that it is being debugged.  The program 
uses ptrace to do so, a common anti-reverse engineering tactic.  While the
majority of my analysis of the code was static, just reading the code and
translating to higher level constructs, I occasionally needed to perform 
tests with the binary.  To avoid having to manually step over the call to 
ptrace in \_start every single time, I patched the binary with gdb to call
main \textit{if it was being debugged}, instead of if it was not.  I did so
with the following commands:

\begin{lstlisting}[language=bash]
cp bomb bomb.patched
gdb -write -q ./bomb.patched
disassemble /r _start
....
0x0000000000401461 <+49>:    75 60    jne    0x4014c3
....
set {unsigned char}0x0000000000401461 = 0x74
disassemble /r _start
....
0x0000000000401461 <+49>:    74 60    je     0x4014c3
....
quit
\end{lstlisting}

Using an x64 assembler I found that the je and jne instructions are the same
size, and only differ by one byte.  We can skip the fake segfault message by
making a patched copy of the bomb binary that will run in a debugger.

\begin{enumerate}
  \item Stage 1: swordfish
  \item Stage 2: jabraham
  \item Stage 3: 1872
  \item Stage 4: 107 214 428 856 1713
  \item Stage 5: ~ 1172 ~
  \item Stage 6: 48 a 48
  \item Stage 7: (press enter)
  \item Stage 8: (run ./stag8.sh in the same directory as the binary, then 
  press enter).
\end{enumerate}

\newpage
\section{Stage 1 Analysis}
\begin{flushleft}
\vspace{.5pc}
\end{flushleft}

\begin{flushleft}
\textbf{Analysis:}
\vspace{.5pc}
\end{flushleft}

\par
Stage 1 is straightforward.  Below is an abbreviated listing of the relevant
disassembled code:
\begin{lstlisting}
mov     esi, 0x4018f8   ; "swordfish"
mov     rdi, rax        ; user input
call    strcmp
test    eax, eax
setz    al
leave
\end{lstlisting}
\par
This stage calls strcmp with user input and the string "swordfish".  If the 
return value is zero (meaning the strings are identical), this stage returns 
non-zero and the program proceeds.

\begin{flushleft}
\textbf{Pseudocode:}
\vspace{.5pc}
\end{flushleft}
\begin{lstlisting}[language=C]
bool stage_1(char *user_input)
{
  return strcmp(user_input, "swordfish") == 0;
}
\end{lstlisting}

\begin{flushleft}
\textbf{Solve Script:}
\vspace{.5pc}
\end{flushleft}
\par
Since there was no math needed to calculate my password for this stage, no 
script was needed.

\newpage
\section{Stage 2 Analysis}
\begin{flushleft}
\vspace{.5pc}
\end{flushleft}

\begin{flushleft}
\textbf{Analysis:}
\vspace{.5pc}
\end{flushleft}

\par
Stage 2 starts with a call to getenv.  This function takes one parameter and 
returns the value of a specified environment variable.
\begin{lstlisting}
mov     edi, 0x401902    ; "USER"
call    getenv
mov     QWORD PTR [rip+0x201264], rax
\end{lstlisting}
\par
The user name is stored in a global variable that is used throughout the 
binary at rip+0x201264.  In my case the user name is "jabraham".  This and the 
saved user input are passed to strcmp, similarly to Stage 1.
\begin{lstlisting}
mov     rax, [rbp-0x8]
mov     rsi, rdx        ; "jabraham"
mov     rdi, rax        ; "jabraham"
call    strcmp
test    eax, eax
setz    al
leave
\end{lstlisting}
\par
After the call to strcmp, the program checks the return value in eax and sets 
al if the zero flag is set.  The stage is passed if the user supplies their 
own username.

\begin{flushleft}
\textbf{Pseudocode:}
\vspace{.5pc}
\end{flushleft}
\begin{lstlisting}[language=C]
bool stage_2(char *user_input)
{
  username = getenv("USER");
  return strcmp(user_input, username) == 0;
}
\end{lstlisting}

\begin{flushleft}
\textbf{Solve Script:}
\vspace{.5pc}
\end{flushleft}
\par
As with Stage 1, there was no need to perform calculations to find the 
password for this stage.

\newpage
\section{Stage 3 Analysis}
\begin{flushleft}
\vspace{.5pc}
\end{flushleft}

\begin{flushleft}
\textbf{Analysis:}
\vspace{.5pc}
\end{flushleft}

\par
Stage 3 takes user input and saves it in [rbp-0x28].  The global variable 
holding the username is moved into [rbp-0x8].
\begin{lstlisting}
push    rbp
mov     rbp, rsp
sub     rsp, 30h
mov     [rbp-0x28], rdi
mov     rax, fs:28h
mov     [rbp-0x08], rax
xor     eax, eax
mov     rcx, cs:src
mov     rax, [rbp-0x28]
\end{lstlisting}
Once these variables are set, they are passed to the C function strncat.  It 
takes three parameters: the destination, the source, and the maximum number of
bytes to concatenate.
\begin{lstlisting}
mov     edx, 0x80       ; number
mov     rsi, rcx        ; source (username)
mov     rdi, rax        ; destination (input)
call    strncat
\end{lstlisting}
\par
The destination will now look similar to this: "100jabraham".  The bomb then 
calls strlen on the original username variable.
\begin{lstlisting}
mov     rax, [rbp+0x28]
mov     rdi, rax        ; username ("jabraham")
call    strlen
\end{lstlisting}
\par
The length of the username is saved on the stack at [rbp-0x10].  Then the bomb
gets ready for a call to sscanf.  This function scans a string using a format
string and stores each match in following parameters.  Here sscanf is passed 
the concatenated string, the format specifier "\%u", and the address of a 
stack variable.  The "\%u" tells us that sscanf is looking for a number, 
meaning the password for this stage is a single number.  
\begin{lstlisting}
mov     [rbp-0x10], rax
mov     [rbp-0x14], 0
lea     rdx, [rbp-0x14]
mov     rax, [rbp-0x28]
mov     esi, 0x401907  ; "%u"
mov     rdi, rax
mov     eax, 0
call    sscanf
\end{lstlisting}
\par
Next the binary performs series of calculations with the total length and 
input number.  Note: for readability, I have changed the offsets to reflect
each stack variable's purpose.
\begin{lstlisting}
mov     eax, [rbp+input_as_a_number]
mov     eax, eax
mov     edx, 0
div     [rbp+input_len]
mov     [rbp+input_as_a_number], eax
mov     eax, [rbp+input_as_a_number]
shr     eax, 2
mov     [rbp+input_as_a_number], eax
mov     eax, [rbp+input_as_a_number]
mov     edx, 0AAAAAAABh
mul     edx
mov     eax, edx
shr     eax, 1
cmp     rax, [rbp+input_len]
setnbe  al
\end{lstlisting}
Let us decode what is happening in this block of code.  The input number is 
moved to eax.  Then it is divided by the total length of the concatenated
string.  The result is stored in the input number variable, and a logical 
shift right 2 is performed on it.  This value is then multiplied by the value
0xAAAAAAAB and shifted right 1.  The calculated value is compared to the 
original input length and al is set to one if it is greater.  To summarize 
the previous in a more readable format, in order to pass this stage your input
must satisfy the following:
\[ \frac{input\_number\div input\_len}{12} > input\_len\]
\par
If so, the function returns non zero and the bomb continues to stage 6.

\begin{flushleft}
\textbf{Pseudocode:}
\vspace{.5pc}
\end{flushleft}
\begin{lstlisting}[language=C]
bool stage_3(char *user_input)
{
  int input_as_a_number;
  int input_len;
  unsigned __int64 v4;
  strncat(user_input, username, 128);
  input_len = strlen(user_input);
  sscanf(user_input, "%u", &input_as_a_number);
  return (input_as_a_number / input_len) / 12 > input_len;
}
\end{lstlisting}

\begin{flushleft}
\textbf{Solve Script:}
\vspace{.5pc}
\end{flushleft}
\par
In order to solve the stage I wrote a script in Python 3 to quickly perform 
the math needed.  The script is called "stage3.py" and is located in this
repository for your convenience.  

\begin{lstlisting}[language=Python]
from os import getenv
username = getenv("USER")      # Calculate the length of the username
usernamelen = len(username)
ans = 0                        # Start at zero

# Loop until the first number is greater than
# the length of the number plus the username length

while True:
    ans += 1                   
    length = len(str(ans)) + usernamelen   
    if (ans // length // 12) > length:
        break

print(ans)
\end{lstlisting}
\par
On the UMBC server, this code gives me the integer value 1872.

\newpage
\section{Stage 4 Analysis}
\begin{flushleft}
\vspace{.5pc}
\end{flushleft}

\begin{flushleft}
\textbf{Analysis:}
\vspace{.5pc}
\end{flushleft}
\par
Stage 4's disassembly has conditionals and jumps, making a little bit more
difficult to read but still straightforward.

\begin{lstlisting}
mov     eax,DWORD PTR ds:0x201170       ; the username
movzx   eax, byte ptr [rax]
movsx   eax, al
mov     [rbp+first_letter_in_username], eax
\end{lstlisting}
stage\_4 starts by saving the first character in the username in a local 
stack variable.  

\begin{lstlisting}
mov     r9, rsi
mov     r8, rcx
mov     rcx, rax
mov     rdx, r10
mov     esi, 0x40190a    ; "%u %u %u %u %u"
mov     eax, 0
call    sscanf
add     rsp, 10h
cmp     eax, 5
je      0x9
\end{lstlisting}

The binary then invokes the sscanf function again, this time with the format 
string "\%u \%u \%u \%u \%u".  This means that we need 5 integers to pass the
stage.  In fact, the binary checks if the function returns 5 and fails if 
not.
\begin{lstlisting}
cmp     [rbp+counter], 4
jbe     0x401023
...
mov     rax, [rbp+counter]
mov     eax, [rbp+rax*4+input_numbers]
cmp     eax, [rbp+first_letter_in_username]
jbe     0x40103d
...
mov     rax, [rbp+counter]
mov     eax, [rbp+rax*4+input_numbers]
add     [rbp+first_letter_in_username], eax
jmp     0x401044
...
add     [rbp+counter], 1  ; continue back at top
\end{lstlisting}
This code shows the path of successful execution of stage\_4.  This can be 
thought of as a for loop that executes five times, over each number in our 
input.  During each pass through the loop, the value is added to the ASCII
value of the first character in the user name.  If the next value is smaller
than the previous, the stage fails.

\begin{lstlisting}
mov     rdi, rax        ; username
call    strlen
mov     rcx, rax
mov     rax, rbx
mov     edx, 0
div     rcx
mov     rax, rdx
test    rax, rax
setz    al
\end{lstlisting}
\par
After the loop, the stage calls strlen on the username and then divides the
last sum by the length.  If the remainder (in rdx) is 0, the stage returns 
non-zero.  The pseudocode below shows an approximation of what is happening 
here:

\begin{flushleft}
\textbf{Pseudocode:}
\vspace{.5pc}
\end{flushleft}
\begin{lstlisting}[language=Python]
for number in input_numbers:
	if number <= first_letter:
		return False
	else:
		first_letter += number
return first_letter % strlen(username) == 0
\end{lstlisting}

\begin{flushleft}
\textbf{Solve Script:}
\vspace{.5pc}
\end{flushleft}
\par
I wrote the following script to solve this challenge:
\begin{lstlisting}[language=Python]
from os import getenv
# Get the length of the username
username = getenv("USER")
length = len(username)
first_letter = ord(username[0])

nums = [first_letter + 1]

for i in range(4):
    nums.append(nums[-1] + (nums[-1] ))

# Add one if the last element is odd
if (nums[-1] % 2 == 0):
    nums[-1] += 1

print(" ".join(str(x) for x in nums))
\end{lstlisting}
\par
This script gives me the following output on the server: 107 214 428 856 1713

\newpage
\section{Stage 5 Analysis}
\begin{flushleft}
\vspace{.5pc}
\end{flushleft}

\begin{flushleft}
\textbf{Analysis:}
\vspace{.5pc}
\end{flushleft}
\par
Stage 5 initially uses calls used many times in earlier stages, so I will 
skip over the minutiae.  It calls sscanf on the user input to retrieve three 
values with the following format string: "\%c \%d \%c".  The stage expects
a letter, followed by an integer, and another character.  There is a call to 
strlen to get the length of the username.  

\begin{lstlisting}
mov     rax, [rbp+index]
cmp     rax, [rbp+username_len]
jb      0x40110a
\end{lstlisting}
\par
This is the main loop of the stage and it terminates when the counter is 
not smaller than the username length.
\begin{lstlisting}
mov    rax, QWORD PTR [rax*8+0x401928]
jmp    rax
\end{lstlisting}
\par
This took a bit of research but these sources put me in the right direction:
\begin{itemize}
	\item https://stackoverflow.com/questions/3012011/switch-case-assembly-level-code
	\item https://stackoverflow.com/questions/9815448/jmp-instruction-hex-code
\end{itemize}
The rest of the code represents a jump table based on the letters in the 
username.  I look through the binary to find the offsets, and it is in 
.rodata.  You can use the objdump tool to examine sections, so I run the 
following command: 
\begin{lstlisting}[language=bash]
objdump -s -j .rodata bomb.patched > objectdumps/.rodata
\end{lstlisting}
I write a python script where I annotate each jump for each value.  It is 
called stage5.py.  Once the code gets to the offset, one of the input 
characters is added to or subtracted from the input number.  The calculation
depends on the ASCII value of each character in the username.  The actual 
calculations are in the stage5.py file.  

\begin{flushleft}
\textbf{Solve Script:}
\vspace{.5pc}
\end{flushleft}
\par

\begin{lstlisting}[language=Python]
from os import getenv
username = getenv("USER")   # Get username
username_vals = [ord(character) for character in username]
input_num = 1000            # Just pick a high number. We will adjust later
master_input = 1000
char1 = ord('~')            # Use tildes because it's the highest printable ascii val (126)
char2 = ord('~')

def math_one(val):
    global input_num
    input_num -= (char1 + val)

def math_two(val):
    global input_num
    input_num -= (val * 2)

def math_three(val):
    global input_num
    input_num -= (val)

def math_four(val):
    global input_num
    input_num += char2 - val

def math_five(val):
    global input_num
    input_num -= (val + char2)

def math_six(val):
    global input_num
    input_num -= (char2)

group_one = [97,101,105,111,117,121,math_one]
group_two = [98,99,100,math_two]
group_three = [102,103,104,math_three]
group_four = [106,107,108,109,110,math_four]
group_five = [112,113,114,115,116,math_five]
group_six = [118,119,120,122,math_six]

master_list = [group_one, group_two, group_three, group_four, group_five, group_six]

for value in username_vals:
    group = [x for x in master_list if value in x]      #find our list
    group[0][-1](value)

ans = master_input - input_num
print("{} {} {}".format(chr(char1), ans, chr(char2)))
\end{lstlisting}
This script gives me "~ 1172 ~" on the server. 

\newpage
\section{Stage 6 Analysis}
\begin{flushleft}
\vspace{.5pc}
\end{flushleft}

\begin{flushleft}
\textbf{Analysis:}
\vspace{.5pc}
\end{flushleft}
\par
Stage 6
\begin{flushleft}
\textbf{Solve Script:}
\vspace{.5pc}
\end{flushleft}
\par

\newpage
\section{Stage 7 Analysis}
\begin{flushleft}
\vspace{.5pc}
\end{flushleft}

\begin{flushleft}
\textbf{Analysis:}
\vspace{.5pc}
\end{flushleft}
\par

\begin{flushleft}
\textbf{Solve Script:}
\vspace{.5pc}
\end{flushleft}
\par

\newpage
\section{Stage 8 Analysis}
\begin{flushleft}
\vspace{.5pc}
\end{flushleft}

\begin{flushleft}
\textbf{Analysis:}
\vspace{.5pc}
\end{flushleft}
\par

\begin{flushleft}
\textbf{Solve Script:}
\vspace{.5pc}
\end{flushleft}
\par

\newpage
\section{Works Cited}
\begin{thebibliography}{9}
\bibitem{latexcompanion} 
Michel Goossens, Frank Mittelbach, and Alexander Samarin. 
\textit{The \LaTeX\ Companion}. 
Addison-Wesley, Reading, Massachusetts, 1993.
 
\bibitem{einstein} 
Albert Einstein. 
\textit{Zur Elektrodynamik bewegter K{\"o}rper}. (German) 
[\textit{On the electrodynamics of moving bodies}]. 
Annalen der Physik, 322(10):891–921, 1905.
 
\bibitem{knuthwebsite} 
Knuth: Computers and Typesetting,
\\\texttt{http://www-cs-faculty.stanford.edu/\~{}uno/abcde.html}
\end{thebibliography}

\end{document}}